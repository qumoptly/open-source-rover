\documentclass{article}
\usepackage[margin=1in]{geometry}
\usepackage{setspace}
\usepackage{graphicx}
\usepackage{subcaption}
\usepackage{amsmath}
\usepackage{color}
\usepackage{hyperref}
\usepackage{multicol}
\usepackage{framed}
\usepackage{xcolor}
\usepackage{wrapfig}
\usepackage{float}
\usepackage{fancyhdr}
\usepackage{verbatim}
\usepackage{textcomp}
\usepackage{colortbl}
\usepackage{array, booktabs, caption}
\usepackage{makecell}
\pagestyle{fancy}
\lfoot{\textbf{Open Source Rover Mechanical Assembly Manual}}
\rfoot{Page \thepage}
\lhead{\textbf{\leftmark}}
\rhead{\textbf{\rightmark}}
\cfoot{}
\renewcommand{\footrulewidth}{1.8pt}
\renewcommand{\headrulewidth}{1.8pt}
\doublespacing
\setlength{\parindent}{1cm}

% Parts list tables
\renewcommand\theadfont{\bfseries}
\newcolumntype{I}{ >{\centering\arraybackslash} m{2cm} }  % part image
\newcolumntype{N}{ >{\centering\arraybackslash} m{3cm} }  % part name
\newcolumntype{Q}{ >{\centering\arraybackslash} m{0.75cm} }  % ref & qty

\begin{document}

\newcommand\partimg{\includegraphics[width=2cm,height=1.25cm,keepaspectratio]}

\title{Open Source Rover: Body Assembly Instructions}
\author{Authors: Michael Cox, Eric Junkins, Olivia Lofaro}

\makeatletter
\def\@maketitle{
\begin{center}
	\makebox[\textwidth][c]{ \includegraphics[width=0.7\paperwidth]{"Pictures/Body/Body title".png}}
	{\Huge \bfseries \sffamily \@title }\\[3ex]
	{\Large \sffamily \@author}\\[3ex]
	\includegraphics[width=.65\linewidth]{"Pictures/Misc/JPL logo".png}
\end{center}}
\makeatother

\maketitle

\noindent {\footnotesize Reference herein to any specific commercial product, process, or service by trade name, trademark, manufacturer, or otherwise, does not constitute or imply its endorsement by the United States Government or the Jet Propulsion Laboratory, California Institute of Technology. \textcopyright  2018 California Institute of Technology. Government sponsorship acknowledged.}


% Introduction
\newpage


\tableofcontents

\newpage

\section{Machining/Fabrication}

\subsection{9x12 Aluminum Plate Drilling}
\begin{table}[H]
  \centering
  \arrayrulecolor{lightgray}
  \sffamily\footnotesize
  \captionsetup{font={sf,bf}}
  \caption{Parts/Tools Necessary}
  \begin{tabular}{|N|Q|Q|I|N|Q|Q|I|}
      \hline
      \thead{Item} & \thead{Ref} & \thead{Qty} & \thead{Image} & \thead{Item} & \thead{Ref} & \thead{Qty} & \thead{Image} \\
      \hline
      9"x12" Aluminum Plate & S35 & 1 & \partimg{../../../images/components/Structural/S35.jpg} & & & & \\ \hline 
  \end{tabular}
\end{table}

Next we need to drill a hole in one of the 9x12 Aluminum plates \textbf{S35} because we will need a hole of just over 0.5 in diameter for the differential pivot mount. There is already a small hole drilled in the location we want to use, but it needs to be widened substantially. Start with the drill \# 23 and drill the hole shown by Figure \ref{Drilling the Al plate}. Repeat this with drill sizes stepping up until you get to a drill of 0.5 in. Take the 0.5 in hollow rod \textbf{S19} and make sure it spins freely in the hole you have created. If it does not, drill the hole slightly larger or sand/file the hole until the rod spins with no resistance.\footnote{The 0.5 in hollow rod must spin \textit{freely} while mounted inside the bearing blocks (See step 2.2 Differential pivot for example). It may help to follow step 2.2 in this document to test if you have enough clearance.}

\begin{figure}[H]
  \centering
  \begin{minipage}[b]{0.45\textwidth}
    \includegraphics[width=\textwidth]{"Pictures/Fabrication/9x12 Plate cut".PNG}
  \end{minipage}
  \hfill
  \begin{minipage}[b]{0.45\textwidth}
    \includegraphics[width=\textwidth]{"Pictures/Fabrication/9x12 Plate cut2".png}
  \end{minipage}
  \caption{Drilling the Aluminum Plate}
  \label{Drilling the Al plate}
\end{figure}

\newpage
\subsection{Laser Cut Parts}

The front and back panel of the rover are designed to be made from laser cut acrylic, as to keep them modular to design different things to mount there as wanted. The 2D cutout files  are the .DXF files  and can be found on the github under Mechanical - Body Assembly - Laser Cut Parts

 \noindent If you do not have access to a lases cutter there is an online service which you can order these from below:

\begin{itemize}
	\item \href{https://www.sculpteo.com}{https://www.sculpteo.com}
\end{itemize}

To get the above parts from Sculpteo, go to Laser cutting and then upload these files (with mm selected as units). Hit Next. Make sure scale is set to 100\%, change the material to Acrylic, have thickness to 1/8 inch, and then select whatever color you wish.


\section{Mechanical/Structural Assembly}

**DISCLAIMER** All images were created using CAD that has \#6-32 Socket head cap screws, but the parts list calls for button head screws. The button head screws are correct. 

\subsection{Chassis}

\begin{table}[H]
    \centering
    \arrayrulecolor{lightgray}
    \sffamily\footnotesize
    \captionsetup{font={sf,bf}}
    \caption{Parts/Tools Necessary}
    \begin{tabular}{|N|Q|Q|I|N|Q|Q|I|}
        \hline
        \thead{Item} & \thead{Ref} & \thead{Qty} & \thead{Image} & \thead{Item} & \thead{Ref} & \thead{Qty} & \thead{Image} \\
        \hline
        Dual Side mount A & S17 & 4 & \partimg{../../../images/components/Structural/S17.jpg} & \#6-32x3/8" Button Head Screw & B2 & 4 & \partimg{../../../images/components/Screws/B2.png} \\ \hline
        4.5"x12" Aluminum Plate & S37 & 2 & \partimg{../../../images/components/Structural/S37.jpg} &  &  &  &  \\ \hline
        1" PVC Clamp & S24 & 1 & \partimg{../../../images/components/Structural/S24.jpg} & Allen Key Set & & & \partimg{../../../images/components/Tools/D2.jpeg} \\ \hline
        9"x12" Aluminum Plate & S35A & 1 & \partimg{../../../images/components/Structural/S35.jpg} & 5/16" Wrench & & & \partimg{../../../images/components/Tools/D1.jpg} \\ \hline 
        \#6-32x1/4" Button Head Screw & B1 & 16 & \partimg{../../../images/components/Screws/B1.png} & & & & \\ \hline
    \end{tabular}
\end{table}

\begin{enumerate}
\item \textbf{Top panel: } Take the modified 9x12 Aluminum plate \textbf{S35A} and attach the four Dual side mount A \textbf{S1} using screws \textbf{B1} at the locations shown below. Take care to match the orientation shown. 

\begin{figure}[H]
  	\centering
  	\begin{minipage}[b]{0.45\textwidth}
    		\includegraphics[width=\textwidth]{"Pictures/Body/body_1".PNG}
  	\end{minipage}
  	\hfill
  	\begin{minipage}[b]{0.45\textwidth}
    		\includegraphics[width=\textwidth]{"Pictures/Body/body_2".PNG}
  	\end{minipage}
  	\caption{Attaching side mounts to top panel}
	\label{top_plate}
\end{figure}

\item \textbf{Attach the side panels: } Attach the 4.5x12 plates \textbf{S37} to the dual side mounts using screws \textbf{B1} 

\begin{figure}[H]
 	\centering
 	\begin{minipage}[b]{0.45\textwidth}
    		\includegraphics[width=\textwidth]{"Pictures/Body/body_3".PNG}
  	\end{minipage}
  	\hfill
  	\begin{minipage}[b]{0.45\textwidth}
    		\includegraphics[width=\textwidth]{"Pictures/Body/body_4".PNG}
  	\end{minipage}
  	\caption{Attach the side panels}
	\label{Body side panels}
\end{figure}

\item \textbf{Attach the PVC clamping hub:} Attach the 1-inch PVC bore clamping hub \textbf{S24} to the top plate of the body using screws \textbf{B1} Use the location shown in Figure \ref{pvc to top plate}.

\begin{figure}[H]
  \centering
  \begin{minipage}[b]{0.40\textwidth}
    \includegraphics[width=\textwidth]{"Pictures/Body/body_pvc_1".PNG}
  \end{minipage}
  \hfill
  \begin{minipage}[b]{0.50\textwidth}
    \includegraphics[width=\textwidth]{"Pictures/Body/body_pvc_2".PNG}
  \end{minipage}
  \caption{Attach the PVC clamp to top plate}
	\label{pvc to top plate}
\end{figure}

\end{enumerate}


\subsection{Differential Pivot Block}
The differential pivot is used to transfer weight off of the wheel that is currently climbing to the other front wheel, allowing the rover to climb more easily. Additionally, it serves as a second contact point for the rover's body such that it does not rotate freely about the cross rod.

\begin{table}[H]
    \centering
    \arrayrulecolor{lightgray}
    \sffamily\footnotesize
    \captionsetup{font={sf,bf}}
    \caption{Parts/Tools Necessary}
    \begin{tabular}{|N|Q|Q|I|N|Q|Q|I|}
        \hline
        \thead{Item} & \thead{Ref} & \thead{Qty} & \thead{Image} & \thead{Item} & \thead{Ref} & \thead{Qty} & \thead{Image} \\
        \hline
        0.5" Pillow Bearing Block & S11 & 2 & \partimg{../../../images/components/Structural/S11.jpg} & \#6-32x1" Button Head Screw & B6 & 4 & \partimg{../../../images/components/Screws/B6.png} \\ \hline
        \#6-32x1/4" Spacer & T1 & 8 & \partimg{../../../images/components/Standoffs/T1.png} & Allen Key Set & D2 & & \partimg{../../../images/components/Tools/D2.jpeg} \\ \hline        
        \#6-32 Hex nuts & B11 & 4 & \partimg{../../../images/components/Screws/B11.png} & & & & \\ \hline
    \end{tabular}
\end{table}

\begin{enumerate}
\item \textbf{Mount the pillow bearing blocks:} Using spacers \textbf{T1}, screws \textbf{B6}, and hex nut \textbf{B11}, mount the pillow blocks \textbf{S11} to the top of the body over the hole in the aluminum plate that you drilled earlier as shown in Figure \ref{mount pillow blocks}.

\begin{figure}[H]
  \centering
  \begin{minipage}[b]{0.30\textwidth}
    \includegraphics[width=\textwidth]{"Pictures/Body/Step 5 a".PNG}
  \end{minipage}
  \hfill
  \begin{minipage}[b]{0.55\textwidth}
    \includegraphics[width=\textwidth]{"Pictures/Body/Step 5 b".PNG}
  \end{minipage}
  \caption{Mounting the pillow blocks}
  \label{mount pillow blocks}
\end{figure}

\end{enumerate}

\subsection{Control Board PCB}

\begin{table}[H]
    \centering
    \arrayrulecolor{lightgray}
    \sffamily\footnotesize
    \captionsetup{font={sf,bf}}
    \caption{Parts/Tools Necessary}
    \begin{tabular}{|N|Q|Q|I|N|Q|Q|I|}
        \hline
        \thead{Item} & \thead{Ref} & \thead{Qty} & \thead{Image} & \thead{Item} & \thead{Ref} & \thead{Qty} & \thead{Image} \\
        \hline
        Assembled Control Board PCB & E1 & 1 & \partimg{../../../images/components/Electronics/E1.PNG} & \#6-32x3/8" Button Head Screw & B2 & 4 & \partimg{../../../images/components/Screws/B1.png} \\ \hline
    \end{tabular}
\end{table}

You may want to skip this step until the PCB is completed and tested and insert the PCB later. 

\begin{enumerate}

\item \textbf{Mount the PCB:} Mount the PCB \textbf{E1} to the top of the chassis using screws \textbf{B2}. Note the position of the PCB below.

\begin{figure}[H]
  \centering
  \begin{minipage}[b]{0.50\textwidth}
    \includegraphics[width=\textwidth]{"Pictures/Body/body_5".PNG}
  \end{minipage}
  \hfill
  \begin{minipage}[b]{0.40\textwidth}
    \includegraphics[width=\textwidth]{"Pictures/Body/body_6".PNG}
  \end{minipage}
  \caption{Mounting the PCB}
  \label{pcb}
\end{figure} 

\begin{figure}[H]
	\centering
	\includegraphics[width=0.45\textwidth]{"Pictures/Body/body_7A".PNG}
  \caption{Mounting the PCB}
  \label{pcb2}
\end{figure}

Note the screw positions in Figure \ref{pcb2}. You might not be able to get the screw circled in light blue in because of the PVC clamping hub. This is okay, install the other 3 circled in yellow.

\end{enumerate}


\subsection{Closing the Body}

\begin{table}[H]
    \centering
    \arrayrulecolor{lightgray}
    \sffamily\footnotesize
    \captionsetup{font={sf,bf}}
    \caption{Parts/Tools Necessary}
    \begin{tabular}{|N|Q|Q|I|N|Q|Q|I|}
        \hline
        \thead{Item} & \thead{Ref} & \thead{Qty} & \thead{Image} & \thead{Item} & \thead{Ref} & \thead{Qty} & \thead{Image} \\
        \hline
        Dual Side Mount A & S17 & 12 & \partimg{../../../images/components/Structural/S17.jpg} & \#6-32x1/4" Button Head Screw & B1 & 28 & \partimg{../../../images/components/Screws/B1.png} \\ \hline
        9"x12" Aluminum Plate & S35 & 1 & \partimg{../../../images/components/Structural/S35.jpg} & Volt Meter & E38 & 1 & \partimg{../../../images/components/Electronics/E38.png} \\ \hline
	      Laser Cut Front Panel & S39 & 1 & \partimg{../../../images/components/Structural/S39.PNG} & Laser Cut Back Panel & S40 & 1 & \partimg{../../../images/components/Structural/S40.PNG} \\ \hline
	      Switch & E39 & 1 & \partimg{../../../images/components/Electronics/E39.png} & 3D printed Battery Holder & S41 & 2 & \partimg{../../../images/components/Structural/S41.png} \\ \hline
	      \# 6-32 Heat set insert & I2 & 4 & \partimg{../../../images/components/Inserts/I2.png} & & & & \\ \hline
    \end{tabular}
\end{table}

\begin{enumerate}

\item \textbf{Attach the Dual Side Mounts:} Mount Dual Side Mounts A \textbf{S17} using screws \textbf{B1} on the side plates in both the front and back. 

\begin{figure}[H]
  \centering
  \begin{minipage}[b]{0.45\textwidth}
    \includegraphics[width=\textwidth]{"Pictures/Body/body_8a".PNG}
  \end{minipage}
  \hfill
  \begin{minipage}[b]{0.45\textwidth}
    \includegraphics[width=\textwidth]{"Pictures/Body/body_8".PNG}
  \end{minipage}
  \caption{Dual Side Mount A locations}
  \label{Dual Side Mounts}
\end{figure}

\item Also attach Dual Side Mounts A \textbf{S17} using \textbf{B1} screws on the top plate, in the front and back

\begin{figure}[H]
  \centering
  \begin{minipage}[b]{0.45\textwidth}
    \includegraphics[width=\textwidth]{"Pictures/Body/body_9".PNG}
  \end{minipage}
  \hfill
  \begin{minipage}[b]{0.45\textwidth}
    \includegraphics[width=\textwidth]{"Pictures/Body/body_10".PNG}
  \end{minipage}
  \caption{Dual Side Mount A (cont)}
  \label{Dual Side Mounts_cont}
\end{figure}


\item \textbf{Front and back panels:} Using screws \textbf{B1} attach the front and back laser cut panels to the chassis. 

\begin{figure}[H]
  \centering
  \begin{minipage}[b]{0.45\textwidth}
    \includegraphics[width=\textwidth]{"Pictures/Body/body_12".PNG}
  \end{minipage}
  \hfill
  \begin{minipage}[b]{0.45\textwidth}
    \includegraphics[width=\textwidth]{"Pictures/Body/body_13".PNG}
  \end{minipage}
  \caption{Front and Back Panels attached}
  \label{finished body}
\end{figure}


\item \textbf{Attach the switch and volt meter:} Take the Volt Meter \textbf{E38} and the Switch \textbf{E39} and attach them to the back panel. Note that the switch goes in from the inside of the body, and the volt meter should snap press fit in from the outside.

\begin{figure}[H]
  \centering
  \begin{minipage}[b]{0.45\textwidth}
    \includegraphics[width=\textwidth]{"Pictures/Body/voltmeter_switch".PNG}
  \end{minipage}
  \hfill
  \begin{minipage}[b]{0.45\textwidth}
    \includegraphics[width=\textwidth]{"Pictures/Body/voltmeter_switch_b".PNG}
  \end{minipage}
  \caption{Volt Meter/Switch Installation}
  \label{voltmeter}
\end{figure}


\item \textbf{Attach the Battery Holder:} Take the 3D printed Battery holder \textbf{S41} and using a solder Iron at 460 degrees F, push the \#6-32 heat-set insert \textbf{I2} into the battery holder as shown below.

\begin{figure}[H]
  \centering
  \begin{minipage}[b]{0.45\textwidth}
    \includegraphics[width=\textwidth]{"Pictures/Body/battery_holder_a".PNG}
  \end{minipage}
  \hfill
  \begin{minipage}[b]{0.45\textwidth}
    \includegraphics[width=\textwidth]{"Pictures/Body/battery_holder_b".PNG}
  \end{minipage}
  \caption{Installing heat set inserts}
\end{figure}

\item \textbf{Install the battery holders:} Mount the Battery holders onto the body assembly using screws \textbf{B1} as shown in Figure \ref{battery}.

\begin{figure}[H]
  \centering
  \begin{minipage}[b]{0.45\textwidth}
    \includegraphics[width=\textwidth]{"Pictures/Body/battery_install_a".PNG}
  \end{minipage}
  \hfill
  \begin{minipage}[b]{0.45\textwidth}
    \includegraphics[width=\textwidth]{"Pictures/Body/battery_install_b".PNG}
  \end{minipage}
  \caption{Battery holder installation}
  \label{battery}
\end{figure}

\item \textbf{Build bottom plate assembly:} Take the 9x12 Aluminum Plate \textbf{S35} and attach 6 dual side mount A \textbf{S17} along the edges as shown in Figure \ref{build_bottom}. Be sure to verify that the orientation of the dual side mounts \textbf{S17} matches the pictures.

\begin{figure}[H]
	\centering
	\includegraphics[width=0.65\textwidth]{"Pictures/Body/body_bottom_1".PNG}
  \caption{Building the Bottom Panel}
  \label{build_bottom}
\end{figure}

\item \textbf{Attach the bottom plate:} Take the bottom plate assembly you constructed in the previous step and attach it to the chassis.

\begin{figure}[H]
	\centering
	\includegraphics[width=0.65\textwidth]{"Pictures/Body/body_bottom".PNG}
	\caption{Attaching the Bottom Panel}
\end{figure}

\end{enumerate}

The Body Assembly is now Complete!

\begin{figure}[H]
  \centering
  \begin{minipage}[b]{0.45\textwidth}
    \includegraphics[width=\textwidth]{"Pictures/Body/finished_body_a".PNG}
  \end{minipage}
  \hfill
  \begin{minipage}[b]{0.45\textwidth}
    \includegraphics[width=\textwidth]{"Pictures/Body/finished_body_b".PNG}
  \end{minipage}
  \caption{Completed Body Assembly}
\end{figure}

\end{document}
